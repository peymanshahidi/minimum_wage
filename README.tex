\documentclass[10pt]{article}
\usepackage[margin=0.5in]{geometry}
\usepackage[T1]{fontenc}
\usepackage[utf8]{inputenc}
\usepackage{lmodern}
\usepackage{hyperref}
\usepackage{xcolor}
\usepackage{tabularx}
\usepackage{booktabs}
\usepackage{longtable}
\usepackage{array}
\usepackage{verbatim}
\usepackage{fancyvrb}
\usepackage{enumitem}
\usepackage{pifont}
\setlength{\parskip}{6pt}
\setlength{\parindent}{0pt}
\usepackage[utf8]{inputenc}
\usepackage[T1]{fontenc}
\usepackage{hyperref}
\usepackage{geometry}
\usepackage{amsmath}
\usepackage{amssymb}
\usepackage{booktabs}
\usepackage{longtable}
\usepackage{array}
\usepackage{enumitem}
\usepackage{verbatim}

% Define custom column types for tabularx:
\newcolumntype{L}{>{\raggedright\arraybackslash}X}
\newcolumntype{T}{>{\ttfamily\small\raggedright\arraybackslash}X}

% Define an environment for tables with 6pt font
\newenvironment{eightpt}{\begingroup\fontsize{8}{13}\selectfont}{\endgroup}

\begin{document}

\section*{Replication Package}
\subsection*{Price Floors and Employer Preferences: Evidence from a Minimum Wage Experiment}
by \textbf{John Horton}

% ========== Section 1 ==========
\vspace{1em}
\section*{1. Overview}

This document describes the organization of code, data, and software associated with ``\href{https://www.aeaweb.org/articles?id=10.1257/aer.20170637}{Price Floors and Employer Preferences: Evidence from a Minimum Wage Experiment}'' (Horton, 2025a).
The replication package for this project is available at the Github repository \href{https://www.github.com/johnjosephhorton/minimum\_wage.git}{johnjosephhorton/minimum\_wage} (Horton, 2025b) as well as the openICPSR repository \href{http://doi.org/10.3886/E208551V1}{208551V1} (Horton, 2025c).

This project is largely self-documenting in the sense that the key recipes for building the project are captured in code. 
The key pieces for replicating this project are: 

\begin{enumerate}
    \item A \href{https://www.github.com/johnjosephhorton/minimum\_wage/blob/main/writeup/Makefile}{Makefile} that gives recipes for how each figure, table, and called-out number is constructed in the paper.
    \item A \href{https://www.github.com/johnjosephhorton/minimum\_wage/blob/main/writeup/Dockerfile}{Dockerfile} that describes the system set-up needed with respect to computing resources and packages.
\end{enumerate}

A comprehensive list of files, programs, and software dependecies is provided in the current document as well for the reader's convenience.

% ========== Section 2 ==========
\vspace{1em}
\section*{2. Folder Structure}

The main directory contains three folders:
\begin{verbatim}
-analysis
-codebooks
-writeup
\end{verbatim}

The \href{https://www.github.com/johnjosephhorton/minimum_wage/blob/main/analysis}{\texttt{analysis}} folder contains 23 R scripts that generate all outputs used in the paper.
The complete list of programs as well as software requirements and dependencies is given in section 4.1.

The \href{https://www.github.com/johnjosephhorton/minimum_wage/blob/main/codebooks}{\texttt{codebooks}} folder contains 10 Markdown files that give the overview of the contents of each dataset used in the analysis (one Markdown file for each dataset).
The complete list of datasets used in this project is given in section 4.2.

The \href{https://www.github.com/johnjosephhorton/minimum_wage/blob/main/writeup}{\texttt{writeup}} folder contains the main writeup of the paper (\href{https://www.github.com/johnjosephhorton/minimum_wage/blob/main/writeup/minimum\_wage.tex}{\texttt{minimum\_wage.tex}}). 
The figures, tables, and parameter files generated during replication will be stored in \texttt{writeup/plots}, \texttt{writeup/tables}, and \texttt{writeup/parameters}, respectively.
In the repositories hosting the replication package, these locations are empty by design.
When the project is built, these folders are populated with the necessary files. 

When the project is building (more on this in section 3), a \texttt{data} folder is automatically created in the main directory and populated with the relevant datasets.
(There is no need to create an empty \texttt{data} folder manually.)

% ========== Section 3 ==========
\vspace{1em}
\section*{3. Replication}

The build process for this project is managed using ``\href{https://www.gnu.org/software/make/manual/make.html}{make}'' -- a tool for automating workflows.
A replicator will obtain a \texttt{.env} file from the author and can reproduce the entire writeup of the paper via a push-button approach using one of three available options brought in section 3.2.

\subsection*{3.1. Makefile Logic}

The \href{https://www.github.com/johnjosephhorton/minimum_wage/blob/main/writeup/Makefile}{Makefile} in the \texttt{writeup} folder specifies the complete list of file dependencies for replication of the project. 
A comprehensive list of all dependencies is provided in section 4 of the current readme as well.
Below I provide a short tutorial on \texttt{make}. 
Familiar readers may skip the tutorial and go to section 3.2 directly.

\subsubsection*{3.1.1. A Short Make Tutorial}
A Makefile lists recipes for how a particular output used in the paper is constructed.
The Makefile entry for the \texttt{first\_stage.pdf} in the \texttt{plots} folder looks like this: 

\begin{verbatim}
plots/first_stage.pdf: ../analysis/first_stage.R  data/df_mw_first.csv
	cd ../analysis && Rscript first_stage.R
\end{verbatim}

Note that the target, or output is \texttt{plots/first\_stage.pdf}.  
After the colon is what this output depends on, code-wise, which is an R file in the \texttt{analysis} folder called \texttt{first\_stage.R}.
The tabbed-in-line below is the recipe for how the target is constructed. 
In this case, it is just by running \texttt{Rscript} on \texttt{first\_stage.R.}

There is some code that is shared across multiple figures or tables in this repository. 
For example, \texttt{utilities\_outcome\_experimental\_plots.R} has helper functions used in several plots.
To capture this dependency, there are entries in the Makefile that look like this: 

\begin{verbatim}
analysis/plot_any_exper.R: analysis/utilities_outcome_experimental_plots.R
	touch analysis/plot_any_exper.R
\end{verbatim}

The command \texttt{touch} here updates the timestamp of \texttt{analysis/plot\_any\_exper.R} which in turn would cause the recipe for \texttt{any\_exper.pdf} to be re-run, as \texttt{plot\_any\_exper.R} is a dependency.

\subsection*{3.2. Building the Project}

To get the data and build the project, you need to obtain a \texttt{.env} file from the author and put it in the \texttt{\~/minimum\_wage/writeup} folder. 
The \texttt{.env} file contains: 1) a private URL to the zipped and encrypted data, and 2) a key to decrypt the data. 
The \texttt{.env} file will look like this:

\begin{verbatim}
GPG_PASSPHRASE='<password>'
project_name="minimum_wage"
DROPBOX_URL="<url of dropbox link hosting the data>"(base)
\end{verbatim}

These values are used in a bash script that fetches the data, at \href{https://www.github.com/johnjosephhorton/minimum_wage/blob/main/fetch_data.sh}{fetch\_data.sh} in the main directory.

There are three options to get the data and build the project: 1) local machine approach, 2) Docker approach, and 3) Replit approach.

\subsubsection*{3.2.1. Non-Docker, Local Approach}

First, assuming the \texttt{.env} file is in your \texttt{Downloads} folder, run the folloiwng in your command line:

\begin{verbatim}
git clone git@github.com:johnjosephhorton/minimum_wage.git
cd minimum_wage
cp ~/Downloads/.env writeup
\end{verbatim}

Equivalently, you can download the project and manually put the \texttt{.env} file in the \texttt{writeup} folder.

On a Linux machine, the \href{https://www.github.com/johnjosephhorton/minimum_wage/blob/main/system_setup.sh}{system\_setup.sh} file in the main directory will install the right dependencies. 
In the command line, type:

\begin{verbatim}
sudo ./system_setup.sh
\end{verbatim}

Then, running the following will build the project:

\begin{verbatim}
cd writeup 
make minimum_wage.pdf
\end{verbatim}

On a non-Linux machine, ensure the R packages listed in section 4.1.2 are installed on your machine. 
Then, run the following to build everything:

\begin{verbatim}
git clone git@github.com:johnjosephhorton/minimum_wage.git
cd minimum_wage/writeup
make minimum_wage.pdf
\end{verbatim}

\subsubsection*{3.2.2. Docker Approach}

As in the Non-Docker approach, clone (or download) the repository and ensure a copy of the \texttt{.env} file is present in the \texttt{writeup} folder.
To clone the repository and copy a version of the \texttt{.env} file to the \texttt{writeup} folder you can type the following in the command line:

\begin{verbatim}
git clone git@github.com:johnjosephhorton/minimum_wage.git
cd minimum_wage/writeup
cp ~/Downloads/.env .
\end{verbatim}

From here, running this will build the entire project:

\begin{verbatim}
make docker
\end{verbatim}

After the project is built, a copy of the final pdf will appear inside the \texttt{writeup} folder on your local machine.

\subsubsection*{3.2.3. Replit Approach}

I have created a public replit ``repl'' here: \href{https://replit.com/@johnhorton/minimumwage}{https://replit.com/@johnhorton/minimumwage}.
You can \textbf{fork} this repository, then add the \texttt{.env} file to the \texttt{writeup} folder then just push the big green `Run' button.
The dependencies are specified in the \href{https://www.github.com/johnjosephhorton/minimum_wage/blob/main/replit.nix}{replit.nix} file available in the main directory.
It will generate the PDF and store it in \texttt{writeup}.

% ========== Section 4 ==========
\vspace{1em}
\section*{4. Code, Datasets, and Outputs}

This section describes software requirements, datasets used, and outputs of the analysis.

\subsection*{4.1 Program and Software}

The entire project has been implemented in R (version 4.4.1). The \texttt{analysis} folder contains 23 R scripts that utilize 10 datasets to generate all outputs reported in the paper. 

\subsubsection*{4.1.1. R Scripts}

Table below provides a list of all 23 R scripts located in the \texttt{analysis} folder.

\begin{eightpt}
\begin{tabularx}{\textwidth}{TT}
\toprule
R Script & R Script \\
\midrule
avg\_wages\_by\_cat.R & first\_stage.R \\
jjh\_misc.R & parameters.R \\
parameters\_country\_selection.R & parameters\_effects.R \\
plot\_any\_exper.R & plot\_application\_event\_study.R \\
plot\_composition.R & plot\_did\_all\_outcomes.R \\
plot\_event\_study\_hired\_admin.R & plot\_event\_study\_hourly\_rate\_hired.R \\
plot\_feedback.R & plot\_fill\_and\_hours.R \\
plot\_follow\_on\_openings.R & plot\_hours\_zero.R \\
plot\_organic\_applications.R & quantile\_hours\_worked.R \\
randomization\_check.R & realized\_wage\_distro.R \\
settings.R & table\_any\_prior.R \\
utilities\_outcome\_experimental\_plots.R & \\
\bottomrule
\end{tabularx}
\end{eightpt}

\subsubsection*{4.1.2. R Packages}

The complete list of R packages used in the scripts above, along with the versions I last used to run the code is given in the table below.

\begin{eightpt}
\begin{tabularx}{\textwidth}{TT}
\toprule
R Package (version) & R Package (version) \\
\midrule
cowplot (1.1.3)          & lmtest (0.9-40)           \\
data.table (1.16.2)      & lubridate (1.9.4)         \\
directlabels (2024.1.21) & magrittr (2.0.3)          \\
dplyr (1.1.4)            & plyr (1.8.9)              \\
dotenv (1.0.3)           & purrr (1.0.2)             \\
ggplot2 (3.5.1)          & quantreg (6.00)           \\
ggrepel (0.9.6)          & sandwich (3.1-1)          \\
gridExtra (2.3)          & scales (1.3.0)            \\
gt (0.11.1)              & stargazer (5.1)           \\
lfe (3.1.1)              & tidyr (1.3.1)             \\
\bottomrule
\end{tabularx}
\end{eightpt}

\subsection*{4.2. Datasets}

The analysis uses 10 different datasets, all of which are automatically stored in the \texttt{data} folder after initiating the data download procedure (more on this later).
Table below gives a list of all datasets along with a brief description of each.
The \texttt{codebooks} folder contains 10 Markdown files that document the contents of each dataset.

\begin{eightpt}
\begin{tabularx}{\textwidth}{T@{}p{0.5\textwidth}p{0.5\textwidth}@{}}
\toprule
Dataset & Description \\
\midrule
{df\_mw\_first.csv} & Main experimental outcomes at job post level (first observation) \\
{df\_mw\_all.csv} & Main experimental outcomes at job post level (all observations) \\
{df\_mw\_admin.csv} & Main experimental outcomes at job post level for admin data \\
{df\_mw\_lpw.csv} & Main experimental outcomes at job post level for low wage positions \\
{df\_exp\_results.csv} & Aggregated experimental results and summary statistics \\
{hires\_country\_composition.csv} & Geographic composition of hires \\
{event\_study\_windows.csv} & Time windows and intervals used in the event study analysis \\
{did\_panel.csv} & Data used for the DiD analysis \\
{event\_study\_hired.csv} & Detailed hiring outcome data for event study analyses \\
{event\_study\_windows\_hr\_v\_fp.csv} & Composition of fixed price and hourly jobs over time \\
\bottomrule
\end{tabularx}
\end{eightpt}


\subsection*{4.3. Outputs}

\subsubsection*{4.3.1. Figures}

Table below provides a detailed mapping between each figure used in the text of the paper, which R script generates it, and its dataset dependencies. 
(Note that the location in paper entries correspond to the published version of the manuscript, not the version produced by the replication package.)
All figures will be stored in \texttt{writeup/plots} after the project is built.

\begin{eightpt}
\begin{tabularx}{\textwidth}{@{}p{0.07\textwidth}p{0.34\textwidth}p{0.40\textwidth}>{\centering\arraybackslash}p{0.125\textwidth}@{}}
\toprule
Figure & R Script & Data Dependency & Location in Paper \\
\midrule
Figure 1 & \texttt{analysis/realized\_wage\_distro.R} & \texttt{data/df\_mw\_first.csv} & Page 125 \\
Figure 2 & \texttt{analysis/first\_stage.R} & \texttt{data/df\_mw\_first.csv} & Page 126 \\
Figure 3 & None (TikZ Diagram) & None & Page 126 \\
Figure 4 & \texttt{analysis/plot\_fill\_and\_hours.R} & \texttt{data/df\_mw\_all.csv, df\_mw\_admin.csv, df\_mw\_lpw.csv} & Page 130 \\
Figure 5 & \texttt{analysis/plot\_composition.R} & \texttt{data/df\_mw\_all.csv, df\_mw\_admin.csv, df\_mw\_lpw.csv} & Page 134 \\
Figure 6 & \texttt{analysis/plot\_event\_study\_hourly\_rate\_hired.R} & \texttt{data/event\_study\_hired.csv} & Page 138 \\
Figure 7 & \texttt{analysis/plot\_did\_all\_outcomes.R} & \texttt{data/did\_panel.csv} & Page 140 \\
Figure 8 & \texttt{analysis/plot\_application\_event\_study.R} & \texttt{data/event\_study\_windows.csv} & Page 143 \\
Figure A1 & \texttt{analysis/plot\_organic\_applications.R} & \texttt{data/df\_mw\_all.csv, df\_mw\_admin.csv, df\_mw\_lpw.csv} & Page A2 \\
Figure A2 & \texttt{analysis/plot\_follow\_on\_openings.R} & \texttt{data/df\_mw\_all.csv, df\_mw\_admin.csv, df\_mw\_lpw.csv} & Page A3 \\
Figure A3 & \texttt{analysis/avg\_wages\_by\_cat.R} & \texttt{data/df\_mw\_first.csv} & Page A6 \\
Figure B1 & \texttt{analysis/plot\_hours\_zero.R} & \texttt{data/df\_mw\_all.csv, df\_mw\_admin.csv, df\_mw\_lpw.csv} & Page A7 \\
Figure B2 & \texttt{analysis/plot\_any\_exper.R} & \texttt{data/df\_mw\_all.csv, df\_mw\_admin.csv, df\_mw\_lpw.csv} & Page A9 \\
Figure B3 & \texttt{analysis/plot\_feedback.R} & \texttt{data/df\_mw\_all.csv, df\_mw\_admin.csv, df\_mw\_lpw.csv} & Page A9 \\
Figure B4 & \texttt{analysis/plot\_event\_study\_hired\_admin.R} & \texttt{data/event\_study\_hired.csv} & Page A11 \\
\bottomrule
\end{tabularx}
\end{eightpt}

\subsubsection*{4.3.2. Tables}

Table below provides a detailed mapping between each table used in the text of the paper, which R script generates it, and its dataset dependencies.
(Note that the location in paper entries correspond to the published version of the manuscript, not the version produced by the replication package.) 
All tables will be stored in \texttt{writeup/tables} after the project is built.

\begin{eightpt}
\begin{tabularx}{\textwidth}{@{}p{0.07\textwidth}p{0.34\textwidth}p{0.40\textwidth}>{\centering\arraybackslash}p{0.125\textwidth}@{}}
\toprule
Table & R Script & Data Dependency & Location in Paper \\
\midrule
Table A1 & \texttt{analysis/randomization\_check.R} & \texttt{data/df\_mw\_first.csv} & Page A2 \\
Table B1 & \texttt{analysis/quantile\_hours\_worked.R} & \texttt{data/event\_study\_windows\_hr\_v\_fp.csv} & Page A8 \\
Table B2 & \texttt{analysis/table\_any\_prior.R} & \texttt{data/df\_mw\_first.csv} & Page A10 \\
\bottomrule
\end{tabularx}
\end{eightpt}

\subsubsection*{4.3.3. Parameters}

Some of the scripts do not generate tables or figures, but instead generate ``parameters.''
These are numbers that are called out in the text of the paper but are generated automatically. 
For example, after the project is built the \texttt{parameters/parameters.tex} file will contain a line 
\verb|\newcommand{\numTotal}{159,656}| that is related to the total allocation to the experiment.
If we look in \texttt{parameters.R} script we see the corresponding code that creates this value.

\begin{verbatim}
addParam("\\numTotal", formatC(dim(df.mw.first)[1], big.mark = ","))
\end{verbatim}

The table below provides the mapping between each parameter file and the R script that generates it. 
All parameter files are saved in \texttt{writeup/parameters} after the corresponding R scripts is run.

\begin{eightpt}
\begin{tabularx}{\textwidth}{@{}LL@{}}
\toprule
Parameter File & R Script \\
\midrule
\texttt{parameters/parameters.tex} & \texttt{analysis/parameters.R} \\
\texttt{parameters/effects\_parameters.tex} & \texttt{analysis/parameters\_effects.R} \\
\texttt{parameters/parameters\_fill\_and\_hours.tex} & \texttt{analysis/plot\_fill\_and\_hours.R} \\
\texttt{parameters/did\_parameters.tex} & \texttt{analysis/plot\_did\_all\_outcomes.R} \\
\texttt{parameters/parameters\_composition.tex} & \texttt{analysis/plot\_composition.R} \\
\texttt{parameters/params\_country\_selection.tex} & \texttt{analysis/parameters\_country\_selection.R} \\
\bottomrule
\end{tabularx}
\end{eightpt}

\subsection*{4.4. License for Code}

The code is licensed under an MIT license.

% ========== Section 5 ==========
\vspace{1em}
\section*{5. Data Availability and Provenance}

\subsection*{5.1. Provenance}

This data comes from a large online labor market (Anonymous Online Platform, 2023) that conducted the experiment described in the paper. 
The data was pulled from the company's database using SQL. 
The data used for this analysis is proprietary and confidential, but may likely be obtained with Data Use Agreements with the data provider.
The data provider itself chooses to remain anonymous.
As such, researchers interested in access to the data may contact John Horton at \texttt{jjhorton@mit.edu}. 
The author will assist with any reasonable replication attempts for two years following publication.

\subsection*{5.2. Statement about Rights}
\begin{itemize}[label=\ding{109}]
  \item [\ding{51}] I certify that the author(s) of the manuscript have legitimate access to and permission to use the data used in this manuscript.
  \item I certify that the author(s) of the manuscript have documented permission to redistribute/publish the data contained within this replication package. Appropriate permissions are documented in the \href{https://www.github.com/johnjosephhorton/minimum_wage/blob/main/LICENSE.txt}{LICENSE.txt} file in the main directory.
\end{itemize}

\subsection*{5.3. Summary of Availability}
\begin{itemize}[label=\ding{109}]
  \item All data \textbf{are} publicly available.
  \item Some data \textbf{cannot be made} publicly available.
  \item [\ding{51}] \textbf{No data can be made} publicly available.
\end{itemize}

% ========== Section 6 ==========
\vspace{1em}
\section*{6. Computational Requirements}

The computational requirements for reproducing this paper are minimal. 
The project can be run on a modern laptop in about 20 minutes, using only open-source softwares.

The code was last run on 2025-03-30 15:16:41. The machine details when I last ran it are:\\
System info:
\begin{itemize}
    \item OS: ``Ubuntu 20.04.6 LTS''
    \item Processor: 11th Gen Intel(R) Core(TM) i7-11850H @ 2.50GHz, 16 cores
    \item Memory available: 31GB memory
\end{itemize}

The Linux dependencies are mostly either for LaTeX or dependencies for various R packages.

\subsection*{6.1. Runtime}

Approximate time needed to reproduce the analyses on a standard 2023 desktop machine is about 20 minutes:
\begin{itemize}[label=\ding{109}]
  \item $<$10 minutes
  \item [\ding{51}] 10--60 minutes
  \item 1--2 hours
  \item 2--8 hours
  \item 8--24 hours
  \item 1--3 days
  \item 3--14 days
  \item $>$14 days
  \item Not feasible to run on a desktop machine, as described below.
\end{itemize}

% ========== Section 7 ==========
\vspace{1em}
\section*{7. References}

\begin{itemize}
  \item Anonymous Online Platform. 2023. ``Production Database on Contractors,'' Unpublished Data, Accessed December 9, 2023.
  \item Horton, John J. 2025a. ``Price Floors and Employer Preferences: Evidence from a Minimum Wage Experiment.'' American Economic Review 115, No. 1: 117--146. \href{https://doi.org/10.1257/aer.20170637}{https://doi.org/10.1257/aer.20170637}.
  \item Horton, John J. 2025b. “minimum\_wage (Version 1.0).” GitHub Repository. Accessed April 9, 2025. \\ \href{https://github.com/johnjosephhorton/minimum_wage/releases/tag/v1.0}{https://github.com/johnjosephhorton/minimum\_wage/releases/tag/v1.0}.
  \item Horton, John J. 2025c. ``Code For: Price Floors and Employer Preferences: Evidence from a Minimum Wage Experiment.'' American Economic Association, Inter-university Consortium for Political and Social Research. \\ \href{http://doi.org/10.3886/E208551V1}{http://doi.org/10.3886/E208551V1}.
\end{itemize}

\end{document}